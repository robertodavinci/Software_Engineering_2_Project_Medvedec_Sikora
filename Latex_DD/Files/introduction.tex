\subsection{Purpose}
\hspace{\parindent}This document provides a detailed view of the architecture and the user interface design of the CLup system. Building on the RASD document, it gives a more refined technical and functional description of the system, explaining it on a much lower level. 

To provide the full description of the system, UML diagrams will be used since they are the de facto industry standard. While the actual implementation is not part of the document, it outlines the presumed implementation, integration, and test plan, to help the software development team in the realization of the application.

The purpose of the document is primarily to guide software developers, but it can also provide useful information to end users and investors.  

\newpage

\subsection{Scope}
\hspace{\parindent}CLup is a simple application that helps store managers with handling large crowds inside their store and store customers with planning more efficient and safe grocery shops. The target audience for this application includes every person that shops for groceries in a store, which includes almost all demographics fall into this category. 

Faced with a worldwide pandemic of the COVID-19 virus countries across the world imposed strict health measures in line with the recommendations of the WHO. To combat the spread of the virus, governments introduced decrees that limited the movement of the population to a certain degree. Only essential movement, such as: going to work, grocery shopping or outdoor exercise, was deemed acceptable. Although successful in the mitigation of the disease, the act put a serious strain on society on many levels. To help reduce the stress and anxiety, many aspects of everyday life involving close contact can be considered and improved upon. 

This project aims to help with, and resolve the issues surrounding grocery shopping. As we all know, grocery shopping is an essential activity which involves close contact inside the store. Since the COVID-19 virus spreads mainly through airborne particles, this activity plays a key role in its mitigation. To reduce crowding inside the stores, supermarkets need to restrict access to their store and keep the number of people inside below the optimal maximum capacity. 

The main idea is to enable store customers to enter a queue from home (or wherever they find themselves) through simple interaction with the application. Besides that, the application will give customers the option to "Book a visit" to the grocery store. This feature will allow them to view available time slots for their grocery shop, book the most convenient one, and optionally indicate an approximated duration of their visit to further improve the accuracy of the waiting time estimation of the system.  

\newpage

\subsection{Definitions, Acronyms, Abbreviations}
\subsubsection{Definitions}
\begin{itemize} 
	\item \textbf{Application}: a computer (mobile) program that is designed for a particular purpose. 
	\item \textbf{QR code}: a machine-readable code consisting of an array of black and white squares, typically used for storing URLs or other information for reading by the camera or a scanner. 
	\item \textbf{Smartphone}: a mobile phone that performs many of the functions of a computer, typically having a touchscreen interface, internet access, and an operating system capable of running downloaded apps. 
	\item \textbf{Google Maps}: a web mapping service developed by Google, used both as a standalone app and as an integrated mapping solution in most of the apps.
	\item \textbf{iOS}: operating system developed by Apple, used by their portable devices like iPads and iPhones.
	\item \textbf{Android}: most popular operating system for smartphones and tablets, developed by Google and partners.
\end{itemize}
\subsubsection{Acronyms}
\begin{itemize}
	\item \textbf{RASD}: Requirement Analysis and Specification Document
	\item \textbf{COVID-19}: Virus responsible for the spread of the coronavirus disease 2019
	\item \textbf{CLup}: Customer Line-up
	\item \textbf{API}: Application programming interface, computing interface which defines interactions between multiple software intermediaries 
	\item \textbf{WHO}: World Health Organization
	\item \textbf{GUI}: Graphical user interface
	\item \textbf{DB}: Database
	\item \textbf{REST}: Representational state transfer - software architectural style used in web services
	\item \textbf{DAO}: Data access object
	\item \textbf{JDBC}: Java Database Connectivity, API used in Java programming language
\end{itemize}
\subsubsection{Abbreviations}
\begin{itemize}
	\item \textbf{Gn}: nth goal.
	\item \textbf{Rn}: nth functional requirement.
	\item \textbf{App}: Application.
\end{itemize}

\newpage
\subsection{Revision History}
\begin{itemize}
	\item \textbf{Version 1.0}: First .tex document created and added all together; 5th January 2021
	\item \textbf{Version 1.1}: Added interface diagram and fixed some errors, minor tweaks; 6th January 2021
	\item \textbf{Version 1.2}: Fixed grammatical errors, added effort spent tables, minor tweaks; 7th January 2021
	\item \textbf{Version 1.3}: Minor tweaks; 9th January 2021
\end{itemize}

\newpage
\subsection{Reference Documents}
\begin{itemize}
	\item Specification document "R\&DD Assignment A.Y. 2020-2021.pdf"
	\item Presentations Software Engineering 2, Politecnico di Milano
	\item Star UML - Program used for creating diagrams
	\item Fundementals of Software Engineering - C. Ghezzi, M. Jazayeri, D. Mandrioli
\end{itemize}


\newpage
\subsection{Document Structure}
\hspace{\parindent} This document is divided into six different chapters, each one further specifying and clarifying the system proposed in the RASD document. \newline

The first chapter starts with the explanation of the purpose of this document and a brief recap of the end product. It also contains formal necessities such as the definitions, acronyms, and abbreviations for better understanding of the matter, the document revision history, the list of referenced documents, and this subchapter, the document structure. \newline 

The second chapter provides a detailed and implementation ready architectural design of the application. Firstly, the overview of the high-level components and their interaction is given, followed by the layout and the definition of the specific components of the system. Furthermore, it explains the deployment and the runtime view, in order to explain the way the components interact and how they relate to the needed use cases. In the end, the communication between the systems components, selected architectural styles and patterns and some other design choices are also explained in detail. \newline

The third chapter revolves around the design of the user interface. To get a sense of what the finished product should look like, this chapter provides an overview on how the user interacts with the system and how the system will look like. This matter was already introduced in the RASD document, and here, expanding that description, the final thoughts and design aspects are presented. \newline

The fourth chapter lists the requirement traceability by mapping requirements, goals, and components. It ties the core ideas of the RASD and the DD document together, by coupling the goals and the requirements listed in the RASD, with specific implementation components defined in the DD. \newline 

The fifth chapter contains our thoughts and guidelines for the actual implementation, integration, and testing of the system. It can serve as a development plan for the software engineers creating an application according to these documents. The chapter identifies the order in which the subcomponents are to be implemented, and at which points in time each one of them should be integrated. In the end, it provides a thorough overview of the test plan of the system as a whole. \newline

The sixth part provides information about the number of hours each group member has spent working on each part of this document. \newline
