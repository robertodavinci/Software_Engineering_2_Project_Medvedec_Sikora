\subsection{Classes, Interfaces, and Enumerations}
\hspace{\parindent} The structure of the code is similar to that explained in both RASD and DD documents. The app communicates with the database through controllers and services, and the database returns the data back. \newline

Since there is no backend on Firebase database, we've changed our design rationale from DD from thin-client to fat-client. This way pretty much all of the work is done by the application, with the database being only used for storing and loading data. Since the operations are very simple, this provides no issues to the functionality of the app. After adding "Book a visit" feature, app would get significantly more demanding and a switch to another database with backend implementation would be welcome. However, this version of the app only takes a few megabytes of space and is very fast even on older phones, with the only decrease of speed being because of the already mentioned Firebase data delay. \newline

One of the main challenges was to sync the synchronous application with the asynchronous database model. This meant making some design decisions that will both look good and feel right when using the app. The app is expected to be fast and consistent, which is something our database can't guarantee, so getting those two in line was a bit of an issue. \newline

Most of the work here is done by using listeners - functions that work on another thread and wait for the data from the database. This allows the app to keep working properly on one thread without waiting for the data. When data arrives, app updates the data that is on the screen. When the wait is too long, that is more than a second, loading screens are introduced to keep the dynamic feel of the app.\newline

The source code division is following:
\begin{itemize}
\item \textbf{Entities}
\begin{itemize}
\item ApplicationState
\item Store
\item StoreManager
\item Ticket
\item TicketState (ennumeration)
\item Timeslot
\item User
\item UserType (ennumeration)
\end{itemize}
\item \textbf{Services}
\begin{itemize}
\item DatabaseManagerService (interface)
\item DirectorService (interface)
\item EnterService (interface)
\item ExitService (interface)
\item LoginManagerService (interface)
\item QueueService (interface)
\item StoreSelectionManagerService (interface)
\item TicketService (interface)
\item \textbf{Implementation}
\begin{itemize}
\item DatabaseManager
\item Director
\item LoginManager
\item RequestManager
\item StoreManager
\item StoreSelectionManager
\end{itemize}
\end{itemize}
\item \textbf{Controllers}
\begin{itemize}
\item CustomerController
\item EncryptionService (not used in this version, encryption has been directly implemented in other classes)
\item ForgotPasswordController
\item HomeController
\item LoginController
\item PreLoginController
\item QrController
\item RegisterController (not used in this version)
\item ScannerController
\item StoreController
\item StoreManagerController
\item StrongAES
\item TicketController
\item UserProfileController (not used in this version)
\end{itemize}
\item \textbf{Listeners}
\begin{itemize}
\item OnCheckTicketListener
\item OnCredentialCheckListener
\item OnGetDataListener
\item OnGetTicketListener
\item OnGetTimeslotListener
\item OnTaskcompleteListener
\end{itemize}
\end{itemize}

All of the files are classes besides the ones that are described differently. Every controller has an additional \textit{activity\_controllername.xml} file that defines the design of an app page on the phone.
\subsection{Code examples}
\hspace{\parindent} Here are provided some code examples for the components. At least one component from each section is provided to give an example how the rest of the components from that section look.\newline

\textbf{Entities - Store}

\begin{lstlisting}
// Store.java
package com.example.clup.Entities;

public class Store {
    public String name, address, city; // variables that define each Store
    public int maxNoCustomers, id; // variables that define each Store

    // Store constructors
    public Store(){}
    public Store(int id, String name, String city){
        this.id = id;
        this.name = name;
        this.city = city;
    }
    public Store(int id, String name, String address, String city){
        this.id = id;
        this.name = name;
        this.address = address;
        this.city = city;
    }
    public Store(int id, String name, String address, String city, int maxNoCustomers){
        this.id = id;
        this.name = name;
        this.address = address;
        this.city = city;
        this.maxNoCustomers = maxNoCustomers;
    }
    // Store getters
    public String getAddress() {
        return address;
    }
    public String getCity() {
        return city;
    }
    public String getName() {
        return name;
    }
    public int getId() {
        return id;
    }
    public int getMaxNoCustomers() {
        return maxNoCustomers;
    }
}
\end{lstlisting}

\textbf{Entities - TicketService}

\begin{lstlisting}
// TicketService.java
package com.example.clup.Services;

import com.example.clup.Entities.Store;
import com.example.clup.Entities.Ticket;
import com.example.clup.OnCheckTicketListener;
import com.example.clup.OnGetDataListener;
import com.example.clup.OnGetTicketListener;
import com.example.clup.OnTaskCompleteListener;

public interface TicketService {
    public void getTicket(Store store, OnGetTicketListener onGetTicketListener);
    public void checkTicket(Ticket ticket, OnCheckTicketListener onCheckTicketListener);
    public void checkQueue(Store store, OnCheckTicketListener onCheckTicketListener);
    public void cancelTicket(Store store, Ticket ticket, OnTaskCompleteListener onTaskCompleteListener);
}
\end{lstlisting}

\textbf{Entities - RequestManager}

\begin{lstlisting}
// RequestManager.java
package com.example.clup.Services.Implementation;

import com.example.clup.Entities.Store;
import com.example.clup.Entities.Ticket;
import com.example.clup.Entities.TicketState;
import com.example.clup.Entities.Timeslot;
import com.example.clup.OnCheckTicketListener;
import com.example.clup.OnGetDataListener;
import com.example.clup.OnGetTicketListener;
import com.example.clup.OnGetTimeslotListener;
import com.example.clup.OnTaskCompleteListener;
import com.example.clup.Services.DatabaseManagerService;
import com.example.clup.Services.QueueService;
import com.example.clup.Services.TicketService;
import com.google.firebase.database.DataSnapshot;
import com.google.firebase.database.DatabaseError;

import java.sql.Time;
import java.sql.Timestamp;
import java.util.ArrayList;
import java.util.List;

public class RequestManager implements QueueService, TicketService {
    private StoreSelectionManager storeSelectionManager;
    private DatabaseManager databaseManager = DatabaseManager.getInstance();

    // Average waiting time and tickets lists have not been implemented in this version, as well as timeslots
    // This can be used as an template for implementing "Book a visit" feature
    List<Ticket> tickets;
    //TODO
    private int averageMinutesInStore = 15, maxId = -1;

    // retrieves Ticket and sets its state based on other Store attributes
    @Override
    public void getTicket(Store store, OnGetTicketListener onGetTicketListener) {
        //System.out.println("Get ticket rm");
        maxId = -1;
        databaseManager.getStore(store, new OnGetDataListener() {
            @Override
            public void onSuccess(DataSnapshot dataSnapshot) {
                if (Integer.parseInt(dataSnapshot.child("open").getValue().toString()) == 0) {
                    onGetTicketListener.onFailure();
                    return;
                }
                maxId = Integer.parseInt(dataSnapshot.child("maxId").getValue().toString());
                Ticket ticket = new Ticket(maxId + 1, store);
                int occupancy = Integer.parseInt(dataSnapshot.child("occupancy").getValue().toString());
                int maxNoCustomers = Integer.parseInt(dataSnapshot.child("maxNoCustomers")
                .getValue().toString());
                int activeTickets = 0;
                for (DataSnapshot i : dataSnapshot.child("Tickets").getChildren()) {
                    if (i.child("ticketState").getValue().toString().equals("ACTIVE"))
                        activeTickets++;
                }
                if (occupancy + activeTickets < maxNoCustomers) {
                    ticket.setTicketState(TicketState.ACTIVE);
                    ticket.setTimeslot(new Timeslot(new Timestamp(System.currentTimeMillis() + 1000 * 60 * 5))); // wait for customer 5 mins
                } else {
                    ticket.setTicketState(TicketState.WAITING);
                    ticket.setTimeslot(new Timeslot(new Timestamp(0)));
                }
                databaseManager.persistTicket(ticket);
                onGetTicketListener.onSuccess(ticket);
            }
            @Override
            public void onFailure(DatabaseError databaseError){
            }
        });
    }

    // checks the Ticket state, whether it's ACTIVE or WAITING
    @Override
    public void checkTicket(Ticket ticket, OnCheckTicketListener onCheckTicketListener) {
        maxId = -1;
        databaseManager.getStore(ticket.getStore(), new OnGetDataListener() {
            @Override
            public void onSuccess(DataSnapshot dataSnapshot) {
                if(dataSnapshot.child("Tickets").
                hasChild(String.valueOf(ticket.getId())) == true) {
                    if(dataSnapshot.child("Tickets")
                    .child(String.valueOf(ticket.getId())).
                    child("ticketState").getValue().toString().
                    equals("ACTIVE")) {
                        //how much does he have left
  onCheckTicketListener.onActive(Timestamp.
  valueOf(dataSnapshot.child("Tickets").child(String.
  valueOf(ticket.getId())).child("expires").getValue().toString()));
                    } else {
                        //calculate people in front
                        int peopleAhead = 1;
                        for (DataSnapshot i : dataSnapshot.child("Tickets").getChildren()) {
                            if
(i.child("ticketState").getValue().toString().equals("WAITING") && Integer.parseInt(i.getKey()) < ticket.getId())
                                peopleAhead++;
                        }
                        onCheckTicketListener.onWaiting(peopleAhead);
                    }
                }
                else {
                    onCheckTicketListener.onBadStore("Ticket has already been used");
                    }
                return;
            }

            @Override
            public void onFailure(DatabaseError databaseError){
                onCheckTicketListener.onBadStore("Bad store information - reload app");
            }
        });
    }
    // checks the current store queue to see how many customers are in line
    @Override
    public void checkQueue(Store store, OnCheckTicketListener onCheckTicketListener) {
        maxId = -1;
        databaseManager.getStore(store, new OnGetDataListener() {
            @Override
            public void onSuccess(DataSnapshot dataSnapshot) {
                //calculate people in front
                if (Integer.parseInt(dataSnapshot.child("open").getValue().toString()) == 0) {
                    onCheckTicketListener.onBadStore("The store is not open");
                    return;
                }
                int peopleAhead = 0;
                for (DataSnapshot i : dataSnapshot.child("Tickets").getChildren()) {
                    if (i.child("ticketState").getValue().toString().equals("WAITING"))
                        peopleAhead++;
                }
                onCheckTicketListener.onWaiting(peopleAhead);
                //System.out.println("AAAAA" + peopleAhead);
                return;
            }
            @Override
            public void onFailure(DatabaseError databaseError){
            }
        });
    }
    // cancels and deletes a Ticket
    @Override
    public void cancelTicket(Store store, Ticket ticket, OnTaskCompleteListener onTaskCompleteListener) {
        databaseManager.getTicket(store, String.valueOf(ticket.getId()), new OnGetDataListener() {
            @Override
            public void onSuccess(DataSnapshot dataSnapshot) {
                if (dataSnapshot.getValue() == null) {
                    onTaskCompleteListener.onFailure(0);
                    return;
                }
                dataSnapshot.getRef().setValue(null);
                onTaskCompleteListener.onSuccess();
                return;
            }
            @Override
            public void onFailure(DatabaseError databaseError){
            }
        });
    }
}

\end{lstlisting}

\textbf{Entities - OnTicketCheckListener}

\begin{lstlisting}
// OnTicketCheckListener.java
package com.example.clup;

import java.sql.Timestamp;

public interface OnCheckTicketListener {
    public void onWaiting(int peopleAhead);
    public void onActive(Timestamp expireTime);
    public void onBadStore(String error);
}

\end{lstlisting}

\textbf{Entities - HomeController}

\begin{lstlisting}
// HomeController.java
package com.example.clup;

import androidx.appcompat.app.AppCompatActivity;
import androidx.core.app.ActivityCompat;
import androidx.core.content.ContextCompat;

import android.Manifest;
import android.content.Context;
import android.content.Intent;
import android.content.SharedPreferences;
import android.content.pm.PackageManager;
import android.os.Bundle;
import android.view.View;
import android.widget.Button;

import com.example.clup.Entities.ApplicationState;
import com.google.firebase.FirebaseApp;
import com.google.firebase.auth.FirebaseAuth;

public class HomeController extends AppCompatActivity implements View.OnClickListener{

    private Button storeButton, loginButton;
    public static final String MyPREFERENCES = "MyPrefs" ;
    private static final int MY_CAMERA_REQUEST_CODE = 100;

    @Override
    protected void onCreate(Bundle savedInstanceState) {
        super.onCreate(savedInstanceState);
        setContentView(R.layout.activity_home_controller);

        storeButton = (Button) findViewById(R.id.storeButton);
        loginButton = (Button) findViewById(R.id.loginButton);
        // Update user
       if (FirebaseAuth.getInstance().getCurrentUser() == null) System.out.println("NOPE");

        // checks for camera permission and asks for it if it's not permitted - scanner will crash the app if
        // the camera is not enabled
        if (checkSelfPermission(Manifest.permission.CAMERA)
                != PackageManager.PERMISSION_GRANTED) requestPermissions(new String[]{Manifest.permission.CAMERA}, MY_CAMERA_REQUEST_CODE);

        storeButton.setOnClickListener(new View.OnClickListener() {
            @Override
            public void onClick(View v) {
                startActivity((new Intent(v.getContext(), StoreController.class)));
            }
        });
        loginButton.setOnClickListener(new View.OnClickListener() {
            @Override
            public void onClick(View v2) {
                if (FirebaseAuth.getInstance().getCurrentUser() == null)
                    startActivity((new Intent(v2.getContext(), LoginController.class)));
                else
                    startActivity((new Intent(v2.getContext(), PreLoginController.class)));
            }
        });
    }

    @Override
    public void onClick(View v) {
    }
    // Sets the action of a back button pressed from Android
    @Override
    public void onBackPressed () {
        ((ApplicationState) getApplication()).clearAppState();
        // Clears stack of activities
        finishAffinity();

    }
}
\end{lstlisting}