\subsection{Purpose}
\hspace{\parindent}This document provides a detailed view of the architecture and the implementation of the CLup system. Based on both RASD and DD documents provided in this project, that can both be found on the above provided GitHub page, this document specifies the development process and used frameworks and philosophies, as well as explains the code structure and design rationale behind the whole system.

The system implementation is done as an Android app that is available for every Android device that supports Android 8.0 and above. 

Detailed code can be found on provided GitHub page and it can also be imported as an Android Studio project in order to make adjustments to the app.


\newpage

\subsection{Scope}
\hspace{\parindent}CLup is a simple application that helps store managers with handling large crowds inside their store and store customers with planning more efficient and safe grocery shops. The target audience for this application includes every person that shops for groceries in a store, so all demographics fall into this category. 

Faced with a worldwide pandemic of the COVID-19 virus countries across the world imposed strict health measures in line with the recommendations of the WHO. To combat the spread of the virus, governments introduced decrees that limited the movement of the population to a certain degree. Only essential movement, such as: going to work, grocery shopping or outdoor exercise, was deemed acceptable. Although successful in the mitigation of the disease, the act put a serious strain on society on many levels. To help reduce the stress and anxiety, many aspects of everyday life involving close contact can be considered and improved upon. 

This project aims to help with, and resolve the issues surrounding grocery shopping. As we all know, grocery shopping is an essential activity which involves close contact inside the store. Since the COVID-19 virus spreads mainly through airborne particles, this activity plays a key role in its mitigation. To reduce crowding inside the stores, supermarkets need to restrict access to their store and keep the number of people inside below the optimal maximum capacity. 

The main idea is to enable store customers to enter a queue from home (or wherever they find themselves) through simple interaction with the application. 

\newpage

\subsection{Definitions, Acronyms, Abbreviations}
\subsubsection{Definitions}
\begin{itemize} 
	\item \textbf{Application}: a computer (mobile) program that is designed for a particular purpose. 
	\item \textbf{QR code}: a machine-readable code consisting of an array of black and white squares, typically used for storing URLs or other information for reading by the camera or a scanner. 
	\item \textbf{Smartphone}: a mobile phone that performs many of the functions of a computer, typically having a touchscreen interface, internet access, and an operating system capable of running downloaded apps. 
	\item \textbf{Google Maps}: a web mapping service developed by Google, used both as a standalone app and as an integrated mapping solution in most of the apps.
	\item \textbf{Android}: most popular operating system for smartphones and tablets, developed by Google and partners.
	\item \textbf{Firebase}: platform created by Google for creating mobile and web applications, can be used as the database.
\end{itemize}
\subsubsection{Acronyms}
\begin{itemize}
	\item \textbf{RASD}: Requirement Analysis and Specification Document
	\item \textbf{COVID-19}: Virus responsible for the spread of the coronavirus disease 2019
	\item \textbf{CLup}: Customer Line-up
	\item \textbf{API}: Application programming interface, computing interface which defines interactions between multiple software intermediaries 
	\item \textbf{WHO}: World Health Organization
	\item \textbf{GUI}: Graphical user interface
	\item \textbf{DB}: Database
	\item \textbf{AES}: Advanced Encryption Standard
\end{itemize}
\subsubsection{Abbreviations}
\begin{itemize}
	\item \textbf{Gn}: nth goal.
	\item \textbf{Rn}: nth functional requirement.
	\item \textbf{App}: Application.
\end{itemize}

\newpage
\subsection{Revision History}
\begin{itemize}
	\item \textbf{Version 1.0}: First .tex document created and added all together; 7th February 2021
\end{itemize}

\newpage
\subsection{Reference Documents}
\begin{itemize}
	\item Specification document "R\&DD Assignment A.Y. 2020-2021.pdf"
	\item Specification document "Implementation Assignment A.Y. 2020-2021.pdf"
	\item Presentations Software Engineering 2, Politecnico di Milano
	\item Star UML - Program used for creating diagrams
	\item Fundementals of Software Engineering - C. Ghezzi, M. Jazayeri, D. Mandrioli
\end{itemize}


\newpage
\subsection{Document Structure}
\hspace{\parindent} The implementation and test deliverable document is divided into six main chapters. The chapters are: Introduction, Requirements, Adopted Development Frameworks, Code Structure, Testing, and Installation Instructions. Each one of them encapsulates a specific aspect of the development of an application based on the RASD and DD documents released earlier. In a few paragraphs the main focuses of each of the chapters is presented.\newline

The first chapter provides an introduction to the matter at hand. Firstly, it recapitulates the purpose of the document and scope of the project. Secondly, it explains the most important definitions, acronyms, and abbreviations. And lastly, it provides an overview of the document revision history, reference documents, and this document as a whole.\newline

The second chapter revolves around the requirements of the project. The chapter introduces the main requirements in mind when the implementation took place and lists them by importance. It explains why each one is important, what are the consequences of its implementation, and argues on the implementation decisions on each and every one of them. A strong focus is also on the relationship between the requirements of the earlier documents vs the requirements of the implementation.\newline

The third chapter gives a thorough overview of the adopted development frameworks. It is further divided based on the programming languages and frameworks, additional algorithms and middleware, and the database model. Each decision is presented with both its advantages and flaws, and the choice is argumented.\newline

The fourth chapter gives a precise account of the code structure of the specific implementation. All used classes, interfaces, and enumerations used in the implementation are listed and explained. Furthermore, some of the more important code samples are also provided.\newline

The fifth chapter revolves around the testing done on the implementation. The testing wad done in three phases: unit testing, UI testing, and security testing. Some inconveniences in testing connected to the former choices are also explained.\newline

The sixth chapter contains the installation instructions of the apk provided with this document.

