The implementation of the desired functionalities of the project described in the RASD and DD meets most of the set goals. 

Overall, the implementation for both the mobile and web client looks and feels professional. Clear, minimalistic, and simple to use, it enables customers to switch to a virtual lineup instead of a physical one without giving it much thought other than installing it on their phone. 

Additionally, the store managers and employees should have no problems whatsoever with the use of the system, as for them, it is equally as clear. The overall experience really is as easy as possible(G.3). 

Moreover, the system does indeed allow the supermarkets to better monitor and limit the access to the store(G.5 and G.7), and precisely know how many people are coming to the store in advance(G.6). The current implementation is, also, ready for simple upgrades that would allow even the customers who do not have access to technology, to physically retrieve a ticket at the store(G.4). The addition of new functionalities such as the waiting time estimation through Google Maps integration and the "book a visit" feature should entail no serious rework of the database and the backend. 

 

It should, however, be noted that the implementation does not, in its current form, meet all goals set in the RASD. The most noteworthy cause of that, is the lack of queue management logic. While the store employee can, with the current version of the system, easily check if the store is currently at capacity and check if the person has a valid ticket("valid" meaning just that it has not yet been used), it is up to him to decide whether to let the new customer in(the system allows new customers to enter even when the store is full). The system offers no support with the influx control. 

Furthermore, the only way a customer can estimate how much time is left until his turn, is to go to the store and check in person. That almost completely beats the purpose of virtual queueing itself, as this way, the main difference between a physical number printing machine and the application is the number retrieval. Although that by itself  does reduce physical contact considerably(G1.3), it does not use the full potential of a well-managed virtual queue. This means that frequent checkups at the store at or near full capacity can result in hazardous situations(G1) and the customer still has to spend considerable time in front of the store(G.2). 

 

In the end, it is important to note that all mentioned drawbacks stem from a small set of design choices on the backend, and as such should be easily changeable and/or upgradeable. The database, the frontend, and most of the backend work as intended and require no changes whatsoever. Overall, the system works robustly and coherently, and is ready for professional deployment, with minimal changes. 