\hspace{\parindent}The implementation of the desired functionalities of the project described in the RASD and DD meets most of the set goals. \newline

Overall, the implementation for both the mobile and web client looks and feels professional. Clean, minimalistic, and simple to use, it enables customers to switch to a virtual lineup instead of a physical one without any major complications other than installing it on their phone. \newline 

Additionally, the store managers and the employees should have no problems whatsoever with the use of the system, as for them, it is equally as clear. The overall experience really is as easy as possible (\textbf{G.3 - Grant customers an overall experience as easy as possible}).\newline

Moreover, the system does indeed allow the supermarkets to better monitor and limit the access to the store (\textbf{G.5 - Allow supermarkets to monitor access to stores in a better way} and \textbf{G.7 - Allow supermarkets to limit the number of access to stores}), and precisely know how many people are coming to the store in advance(\textbf{G.6 - Allow supermarkets to know in advance how many people are coming to stores}). The current implementation is also ready for simple upgrades that would allow even the customers who do not have access to the technology to physically retrieve a ticket at the store(\textbf{G.4 - Allow  customers,  even  the  ones  who  don’t  have  access  to  technology,  to  enjoy  the service}). The addition of new functionalities such as waiting time estimation through Google Maps integration and "Book a visit" feature should entail no serious rework of the database and the backend.\newline


It should, however, be noted that the implementation does not in its current form meet all the goals set in RASD. The most noteworthy cause of that is the lack of queue management logic. While the store employee can with the current version of the system easily check if the store is currently at capacity and check if the person has a valid ticket ("valid" meaning just that it has not yet been used), it is up to him to decide whether to let the new customer in (the system allows new customers to enter even when the store is full). The system offers no support with the influx control.\newline

Furthermore, the only way a customer can estimate how much time is left until his turn is to go to the store and check in person. That almost completely beats the purpose of virtual queueing itself, as this way the main difference between a physical number printing machine and the application is the number retrieval. Although that by itself  does reduce physical contact considerably (\textbf{G1.3 - Shorten the amount of time a customer is in queue}), it does not use the full potential of a well-managed virtual queue. This means that frequent checkups at the store at or near full capacity can result in hazardous situations(\textbf{G1.1 - Allow customers to avoid the creation of hazardous situations}) and the customer still has to spend considerable time in front of the store (\textbf{G.2 - Grant customers an overall experience as easy as possible}). Another thing to notice is that the store employee who is scanning the tickets has to have two devices near him at all times - on scanning device (mobile phone) and one monitoring device (a desktop) which shows him how many people are currently in the store and in queue. If the employee hasn't got a desktop near him, there is no way of telling how many places are still free inside the store, since the mobile app doesn't show that information on the screen, which is a small detail but one that could be very problematic for some stores. \newline

In the end, it is important to note that all mentioned drawbacks stem from a small set of design choices on the backend and as such should be easily changeable and/or upgradeable. The database, the frontend, and most of the backend work as intended and require no changes whatsoever.\newline

Overall, the system works robustly and coherently, and is ready for professional deployment, with minimal changes. 